\chapter*{Введение}						% Заголовок
\addcontentsline{toc}{chapter}{Введение}	% Добавляем его в оглавление

\newcommand{\actuality}{}
\newcommand{\progress}{}
\newcommand{\aim}{{\textbf\aimTXT}}
\newcommand{\tasks}{\textbf{\tasksTXT}}
\newcommand{\novelty}{\textbf{\noveltyTXT}}
\newcommand{\influence}{\textbf{\influenceTXT}}
\newcommand{\methods}{\textbf{\methodsTXT}}
\newcommand{\defpositions}{\textbf{\defpositionsTXT}}
\newcommand{\reliability}{\textbf{\reliabilityTXT}}
\newcommand{\probation}{\textbf{\probationTXT}}
\newcommand{\contribution}{\textbf{\contributionTXT}}
\newcommand{\publications}{\textbf{\publicationsTXT}}

Использование роботизированных систем в производстве позволило многократно увеличить скорость производства, а также сэкономить на содержании штата. Аналогичные преимущества можно было бы получить, используя роботов для обслуживания клиентов заведений общественного питания. На текущий момент эта сфера еще полностью, за исключением двух случаев, свободна.

При разработке алгоритмической части можно пойти двумя путями: реализовать все алгоритмы с нуля самому или же использовать готовые фреймфорки, такие как \textit{ROS}. В первом случае вы получаете глубокое понимание того, что происходит, но маловероятно, что алгоритмы будут реализованы эффективнее, чем аналогичные во фреймворках. Второй же подход избавляет вас от необходимости изучать, как устроены алгоритмы, и позволяет вам оперировать более высокоуровневыми операциями(наподобие "переместить рабочий элемент в точку $\{x, y, z\}^{T}$") с самого начала работы. Оба навыка очень полезны. Имея понимание алгоритмов, вы можете оптимизировать их под вашу конкретную задачу, что делает ваш продукт уникальным. Умея же работать с современными фреймворками дает возможность быстро и безболезненно решать поставленные задачи. Обладая же умениями в обоих сферах, вы можете оптимизировать лишь отдельные алгоритмы фреймворка под себя, тем самым находя золотую середину между скоростью разработки и эффективностью алгоритмов. Поэтому все используемые алгоритмы будут реализованы и проверены в среде \textit{VREP}, а аналогичная им система будет промоделирована в \textit{Gazebo} с использованием \textit{ROS}. В качестве управляемого манипулятора взят робот \textit{UR10} фирмы \textit{Universal Robotics}.

Поскольку система должна иметь возможность встраиваться в любые помещения, предлагается разработать механизм генерирования стандартизированных карт помещений. Они будут включать в себя столы и ячейки стандартизированного размера. Таким образом работоспособность системы не будет нарушена при изменении помещения, а также отпадает надобность в использовании дополнительных сенсоров для построения карты.
Архитектура системы управления сильно зависит от поставленной задачи. В случае робота, способного формировать заказы и выдавать их клиенту существует ряд ограничений:
\begin{itemize}
	\item Окружающее пространство содержит препятствия - мебель. Это означает, что должен быть механизм построения геометрического пути с обходом препятствий. Эта задача решается алгоритмами поиска пути.
	\item При переносе напитков есть риск перевернуть емкость или выплеснуть часть жидкости из нее. Таким образом требуется ограничить максимальные значения ускорений вдоль траектории переноса. Это решается интерполяцией пути, а также параметризацией траекторий. 
	\item Также при переносе подноса с заказом должны сохраняться постоянными углы крена и тангажа, чтобы не перевернуть содержимое. Таким образом решение задачи обратной кинематики требует выбора одного конкретного решения с фиксированными углами, а задача поиска пути должна учитывать ориентацию на каждой итерации.
\end{itemize}

Еще остается выбор пространства, в котором будет происходить управления: joint space или же operational space. Первый основывается на решении задачи обратной кинематики, а второй - на свойстве Якобиана $Q = JF$. Первый подход несколько проще в разработке, поэтому решено выбрать его.

Таким образом система управления должна включать в себя следующие элементы:
\begin{itemize}
	\item Решение прямой кинематики. Требуется для вычисления положения рабочего элемента и проверки на коллизии. См. \ref{ch:2}.
	\item Решение обратной кинематики. Требуется для того, чтобы знать в какие точки в пространстве обощенных координат требуется поместить джойнты. См. \ref{ch:3}.
	\item Планировщик геометрических путей. Требуется для построения пути избегающего препятствия, а также выполнения ограничений на ориентацию рабочего элемента. См. \ref{ch:4}.
	\item Планировщик траекторий. Необходим для выполнения ограничений на плавность передвижения. См. \ref{ch:5}.
\end{itemize}

%% на случай ошибок оставляю исходный кусок на месте, закомментированным
%%Полный объём диссертации составляет  \ref*{TotPages}~страницу с~\totalfigures{}~рисунками и~\totaltables{}~таблицами. %%Список литературы содержит \total{citenum}~наименований.
%
% Полный объём диссертации составляет
% \formbytotal{TotPages}{страниц}{у}{ы}{}, включая
% \formbytotal{totalcount@figure}{рисун}{ок}{ка}{ков} и
% \formbytotal{totalcount@table}{таблиц}{у}{ы}{}.   Список литературы содержит  
% \formbytotal{citenum}{наименован}{ие}{ия}{ий}.
