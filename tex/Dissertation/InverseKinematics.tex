\chapter{Задача обратной кинематики} \label{ch:4}


\section{Введение} \label{sect:4_1}
Очень часто при решении задачи управления манипулятором возникает задача определения углов поворотов звеньев, соответствующих заданному положению. Она носит название "задача обратной кинематики". Кроме того данная проблема также возникает в компьютерной графике. Существует большое количество способов ее решения:
\begin{itemize}
	\item Использование транспонированного якобиана(Jacobian transpose)
	\item Использование псевдо-обратного якобиана(pseudoinverse Jacobian)
	\item Damped least squares
	\item Координатный спуск(Cyclic coordinate descent)
	\item Аналитическое решение
\end{itemize}

В работе рассмотрены все самые популярные методы решения задачи.

\section{Откуда у якобиана ноги растут} \label{sect:4_2}
Как мы могли заметить якобиан достаточно часто встречается при решении задачи обратной кинематики. Это объясняется его крайне полезными свойствами, о которых мы поговорим ниже.

Формально якобиан определяется как
\begin{align*}
	J = \frac{\partial x}{\partial q}
\end{align*} 

Согласно правилам дифференцирования сложных функций:
\begin{align*}
	J = \frac{\partial x}{\partial t} \frac{\partial t}{\partial q}		\rightarrow		\frac{\partial x}{\partial t} = J\frac{\partial q}{\partial t}
\end{align*}

Или:
\begin{align*}
	\dot{x} = J\dot{q}
\end{align*}
То есть якобиан связывает скорости в operational space со скоростями в joint space. Разрешив уравнение относительно $\dot{q}$, получим:
\begin{align} \label{eq:4_2_1}
	\dot{q} = J^{-1}\dot{x}
\end{align}

Уравнение \ref{eq:4_2_1} приводит нас к итеративному способу решения задачи обратной кинематики: заменив скорости на небольшие по своей магнитуде смещения, мы можем итеративно решить задачу. 
Пусть $\vec{e} = \vec{t} - \vec{s}$ - вектор равный требуемому перемещению рабочего устройства нашего манипулятора, $\vec{t}$ - требуемое положение, и $\vec{s}$ - исходное положение. Тогда задавая смещение на каждой итерации, как:
\begin{align}
	\Delta q = \alpha J^{-1} \Delta s  \text{, где $\alpha << 1$ - скаляр, $\Delta s$ - перемещение в operational space,}\\
	\text{ а $\Delta q$ - перемещение в joint space}.
\end{align}
мы в конечном итоге придем к решению. Но существует одна проблема - якобиан очень редко является обратимой матрицей, поэтому требуется способ аппроксимации $J^{-1}$. Далее будут рассмотрены самые популярные методы решения этой проблемы.

\section{Транспонирование якобиана}

 
Поскольку якобиан далеко не всегда является квадратной матрицей и, как следствие, необратим, то возникает  
