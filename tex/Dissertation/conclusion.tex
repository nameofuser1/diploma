\chapter*{Заключение}						% Заголовок
\addcontentsline{toc}{chapter}{Заключение}	% Добавляем его в оглавление

%% Согласно ГОСТ Р 7.0.11-2011:
%% 5.3.3 В заключении диссертации излагают итоги выполненного исследования, рекомендации, перспективы дальнейшей разработки темы.
%% 9.2.3 В заключении автореферата диссертации излагают итоги данного исследования, рекомендации и перспективы дальнейшей разработки темы.
%% Поэтому имеет смысл сделать эту часть общей и загрузить из одного файла в автореферат и в диссертацию:

Основные результаты работы заключаются в следующем.
\begin{enumerate}
	\item Исследованы существующие алгоритмы решения задачи прямой кинематики.
	\item Исследованы существующие алгоритмы решения задачи обратной кинематики.
	\item Исследованы существующие алгоритмы задачи поиска геометрических путей.
	\item Исследованы существующие алгоритмы планирования траектории между двумя точками, а также планирования по полученному геометрическому пути.
	\item На основе разобранных алгоритмов была построена система управления.
	\item Полученная система была реализована с помощью фреймворка ROS и среды моделирования Gazebo.
	\item Был реализован алгоритм автоматической генерации карты помещения, в котором оперирует робот.
\end{enumerate}

В заключение автор выражает благодарность и большую признательность научному руководителю
Куприянову Д.В. за поддержку, помощь, обсуждение результатов и научное руководство.
